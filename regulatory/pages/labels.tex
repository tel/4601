\subsection{Proposed Labeling}

The device labeling will include 4 general categories: packaging,
marketing copy, and operation manuals. Due to the exception of section
502(f) as orchestrated by part 801.125 the operations manual will be
distributed electronically.

\subsubsection{Packaging}
The device is marketed at an operational research lab and therefore
does not require intensive package marketing. Instead an austere
recycled carboard box tied with twine is labeled simply with the
device name (eyeScan Mezzo) and appropriate device IDs.

\subsubsection{Marketing Copy}
The device is marketed at 

\subsection{Operations Manual}
The operations manual will consist of a small paper booklet divided into three subsections. The first will detail the process for making the solution in which the eyeball is suspended; the second will describe how to suspend the eyeball in solution; and the third will consist of instructions for connecting the entire device to a computer. 

The suspension solution is polydimethylsiloxane (PDMS), to be made using the Sylgard 184 Silicone Elastomer Kit. 1 gram of base solution and 0.03 grams of curing agent are to be combined in an optical glass tube at room temperature using a pipette. The resulting solution is to be well mixed.

To suspend the eyeball in solution, the eyeball will to be rinsed using aqueous water. Inverted tweezers will be used to pick up the eyeball by the optic nerve and gently suspend it in the center of the tube containing polymer solution. The tube will then be capped tightly and placed into the top slot of the scanning machine. 

The device’s user interface will consist of two wires. One wire will plug into a standard United States outlet of 120 V at 60 Hz  and the other will be a USB cable that plugs into a standard USB drive in a computer. This will allow the device to transmit scanning data to a computer for  three-dimensional reconstruction. 

%%% Local Variables: 
%%% mode: latex
%%% TeX-master: "../reg"
%%% End: 
