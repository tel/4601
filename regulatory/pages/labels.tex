\subsection{Proposed Labeling}

The device labeling will include three general categories: packaging,
marketing copy, and operation manuals. Due to the exception of section
502(f) as orchestrated by part 801.125, the operations manual will be
distributed electronically.

\subsubsection{Packaging}
The device is marketed at an operational research lab and therefore
does not require intensive package marketing. Instead an austere
recycled carboard box tied with twine is labeled simply with the
device name (eyeScan Mezzo) and appropriate device IDs.

\subsubsection{Operations Manual}
The operations manual will consist of an easily accessible electronic PDF document divided
into three subsections. The first will detail the process for making
the solution in which the eyeball is suspended; the second will
describe how to suspend the eyeball in solution; and the third will
consist of instructions for connecting the entire device to a
computer.

The suspension solution is polydimethylsiloxane (PDMS), to be made
using the Sylgard 184 Silicone Elastomer Kit included with the device. 1 gram of base solution
and 0.03 grams of curing agent are to be combined in an optical glass
tube at room temperature using a pipette. The resulting solution is to
be well mixed.

To suspend the eyeball in solution, the eyeball will be rinsed
using aqueous water. Inverted tweezers will be used to pick up the
eyeball by the optic nerve and gently suspend it in the center of the
tube containing polymer solution. The tube will then be capped tightly
and placed into the top slot of the scanning machine.

The device’s user interface will consist of two wires. One wire will
plug into a standard United States outlet of 120 V at 60 Hz and the
other will be a USB cable that plugs into a standard USB drive in a
computer. This will allow the device to transmit scanning data to a
computer for three-dimensional reconstruction. The device will also
have a power on/off switch at the back.

The Sylgard 184 Silicone Elastomer kit will include a brief operations
manual along with MSDS information in paper format.

\subsubsection{Marketing Copy}

\paragraph{Description}
The device, named eyeScan Mezzo, is a bench-sized digitizer able to  and intended to accurately measure and create digital meshes corresponding to the three-dimensional surface structure of dissected rat eyes. It is accurate to a 2 um resolution. EyeScan Mezzo is not intended for use in the diagnosis or treatment of diseases for live animals or humans. Device testing is conducted using developer generated testing protocol and will meet specified acceptance criteria before the device is marketed. Full testing of the device's EMC characteristics is performed according to IEC 60601-1-2 Part I.

\paragraph{Safety and Effectiveness Considerations}

EyeScan Mezzo is intended to be operated only by trained laboratory personnel in ophthalmological research settings only. It poses no known immediate health hazard to the operator. 

\paragraph{Contraindications, Warnings, Precautions}

PDMS is toxic to humans, and as such, contact with skin and organs should be avoided. If PDMS does come into contact with the skin, eyes, or body, one should immediately clean off PDMS from body part using a paper towel and rinse contaminated area under running water. Refer to the MSDS for further instruction.

Insertion of objects other than that indicated in the operations manual may damage the device.
If any foreign object enters the device, the operator should turn off the device and cut off the power supply to the device immediately. The operator should then remove the protection casing of the device, remove the foreign object, and reinstall the protection casing. If the device is damaged to the point of malfunction, the operator should contact the manufacturer for repairs. 

A liquid spill inside the machine may damage electronic components and pose an electrical hazard.
If a liquid spill occurs, the operator should turn off the device and cut off the power supply to the device immediately. The operator should then send in a request to the manufacturer to replace damaged parts.

Proper laboratory attire such as latex gloves, laboratory coats, and goggles should be worn before operating the device. 

Injury may result if the device is operated by untrained personnel. 


%%% Local Variables: 
%%% mode: latex
%%% TeX-master: "../reg"
%%% End: 
