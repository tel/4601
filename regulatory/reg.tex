\documentclass{article}
\usepackage{../tex/mysty}
\usepackage[final]{pdfpages}
\begin{document}


\maketitlepage{U.S. Food and Drug Administration: Regulatory
  Report}{Sanjay Challa}

\setcounter{tocdepth}{3}
\tableofcontents
\newpage

\section*{Executive summary}
\label{sec:exec-summary}

\section{Regulation Overview}
\label{sec:test-administration}

Medical devices are classified into Class I, II, and III based on the
level of regulatory control required by the FDA. All devices must
comply with the Class I General Controls. These include annual
establishment registration, or registration of owners and operators of
facilities that are involved in the production and distribution of
medical devices, as well as a medical device listing of all products
produced at a certain establishment. Devices must be made in
accordance with current good manufacturing practices (cGMPs), and
manufacturers must develop a system of quality control that in line
with any FDA quality systems (QS) guidelines applicable to their
device. All manufacturing processes and procedures must be
documented. The labeling on the device, including the claims and
advertising, should be in accordance with FDA regulations; it should
be clear, should not be misleading, and should not omit any required
important warnings. Finally, as required by FDA Medical Device
Reporting regulations, medical device manufacturers must report to the
FDA any complaints of malfunction of medical devices as well as
serious injuries or deaths associated with medical devices.

Class II products are defined as products for which general controls
alone are insufficient to assure safety and effectiveness.” Orthopedic
and craniofacial implants are Class II products, as opposed to manual
wheelchairs and hand-held surgical instruments, which are Class I
products.  In addition to following Class I General Controls, Class II
products are subject to Standards or Guidance Documents such as ASTM
or ISO Standards or OCD and OCER Guidance.

For Class II products, manufacturers or companies must submit a
premarket notification in the form of a 510(k). A 510(k) submission
requires the device name, the establishment registration number, the
device classification, documentation of performance standards, and
proposed labels and advertisements, as described in Class I General
Controls. In addition, a 510(k) submission requires a description of
“predicate devices” – devices already in the market that are similar
to the proposed product – and a statement detailing the similarities
and differences between predicate devices and the proposed
device. 510(k) clearance is based on substantial equivalency, which
means that “the new device is at least as safe and effective as the
predicate.” Predicate devices may be in any of the three device
classes. A product is substantially equivalent if it has the same
intended use and technological characteristics as the predicate
device. If a device has the same intended use as the predicate but
different technological characteristics, it should be proven to be as
safe and effective as the predicate and should not raise any new
safety issues.

Class III products are defined as those which “support or sustain
human life, are of substantial importance in preventing impairment of
human health, or which present a potential, unreasonable risk of
illness or injury.” A Pre-Market Approval (PMA) submission and, in
some cases, an investigational device exemption (IDE) are required for
approval of Class III devices. (An IDE may also be required for some
Class II product clearances). An IDE allows the investigational device
to be used in a clinical study, approved both by the FDA and an
Institutional Review Board, to collect safety and effectiveness data
required for a PMA.  In contrast to Class II devices, Class III
devices must receive FDA approval instead of clearance for their use,
components, and manufacturing methods. Class III devices are
significantly different from devices already existent on the market
and present new safety concerns that have not been tested. PMAs must
not only comply with Class I General Controls, but must also
demonstrate the safety and effectiveness of the device and include
both good science and scientific writing.
	 

\section{Device Classification}
\label{sec:protocols}

Searches in the \textit{FDA Product Classification Database} using the
keywords ``tomography,'' ``ophthalmic,'' and ``micrometer'' were
performed. A few entrys were returned which related to our device:

\begin{enumerate}
\item Optical coherence tomography: General \& Plastic Surgery panel,
  regulation number 886.1570, product code OBO; describes an
  ophthalmoscope as an "Diagnostic device to aid in the detection and
  management of various ocular diseases" capable of "viewing, imaging,
  measurement, and analysis of ocular structures."
\item System, x-ray, tomography, computed: Radiology panel, regulation
  number 892.1750, product code JAK; describes a diagnostic x-ray
  system which relies upon similar technical principles as our
  device. The similarity lies in the method by which measurement
  slices are taken, and the manner in which measurement slices are
  combined to create a reconstruction.
\item Micrometers, Microscope: Pathology panel, regulation number
  864.3600, product code KEH; describes a microscopes and accessories
  as optical instruments used to enlarge images of specimens,
  preparations, and cultures for medical purposes.
\end{enumerate}

While the devices listed above share some common elements with our
device, none of the devices fully capture the intended use and
functionality of our device. One key difference between our device and
those listed above is that our device is not intended for any
diagnotic or clinical applications, but is instead targeted at
research applications. In this regard, our device is most likely not
regulated by the FDA as medical device. According to the FD\&C Act
21USC321, a "device" is "an instrument, apparatus, implement, machine,
contrivance, implant, in vitro reagent, or other similar or related
article, including any component, part, or accessory, which is:

\begin{itemize}
\item recognized in the official National Formulary, or the United
  States Pharmacopeia, or any supplement to them,
\item intended for use in the diagnosis of disease or other
  conditions, or in the cure, mitigation, treatment, or prevention of
  disease, in man or other animals, or
\item intended to affect the structure or any function of the body of
  man or other animals, and which does not achieve its primary
  intended purposes through chemical action within or on the body of
  man or other animals and which is not dependent upon being
  metabolized for the achievement of its primary intended purposes."
\end{itemize}

Our device clearly does not fall within any of the elements of the
above definition. It is not listed in the National Formulary, US
Pharmacopeia, nor any supplements to them. Furthermore, our device is
not intended for the diagnosis of disease or other conditions, or in
the cure, mitigation, treatment, or prevention of disease, in man or
other animals. It is solely intended for the measurement of eyeballs
dissected out of mice in research settings. Lastly, our device is not
inteneded for and does not affect the structure or functionality of
any man or animal. 

To conclude, an item by item search in Title 21 of the Code of Federal
Regulations (CFR), part 886 (ophthalmic panel), part 864 (pathology
panel), and part 892 (radiological panel) confirms that our device is
not substantially equivalent to any of the existing devices. While our
device has similar technological and/or functional characteristics, it
differs significantly, as described above. However, considering the
presence of electronic and visible radiation emitting components in
our device, regulations in the FD\&C Act sections 531-542 Electronic
Product Radiation Control provisions are still applicable to our
device.

\refstepcounter{section}
\addcontentsline{toc}{section}{\thesection
  \hspace{1em}510(k) Premarket Notification}
% Cover sheets
\newpage
\addcontentsline{toc}{subsection}{CDRH Premarket Review Submission Cover Sheet}

\begin{figure}[H]
  \centering
  \includegraphics[width=\linewidth]{pages/cdrh-pics/1}
  \label{fig:summary}
\end{figure}

\begin{figure}[H]
  \centering
  \includegraphics[width=\linewidth]{pages/cdrh-pics/2}
  \label{fig:summary}
\end{figure}

\begin{figure}[H]
  \centering
  \includegraphics[width=\linewidth]{pages/cdrh-pics/3}
  \label{fig:summary}
\end{figure}

\begin{figure}[H]
  \centering
  \includegraphics[width=\linewidth]{pages/cdrh-pics/4}
  \label{fig:summary}
\end{figure}

\begin{figure}[H]
  \centering
  \includegraphics[width=\linewidth]{pages/cdrh-pics/5}
  \label{fig:summary}
\end{figure}

%%% Local Variables: 
%%% mode: latex
%%% TeX-master: "../reg"
%%% End: 

\newpage
\singlespacing

\begin{flushright}
  \huge{Traditional 510(k) Submission}\\[.5in]
  
  \begin{minipage}{0.8\textwidth}
    \begin{flushright}
      \large \textbf{Scanner: [[[[[TRADE NAME GOES HERE]]]]]} \\
      \textit{\today}
    \end{flushright}
  \end{minipage}
\end{flushright}

\begin{flushleft}
  Joseph Abrahamson $<$abrahamson.j@gatech.edu$>$ \\
  Phone: 770 601 4606 \\[1em]
  
  \textit{Will register establishment following FDA clearance.}
\end{flushleft}

We seek FDA clearance to market our device, \textbf{TRADE NAME GOES
  HERE}, a Class II medical device used for medical scanning of
three-dimensional exterior eye shape.

%%% Local Variables: 
%%% mode: latex
%%% TeX-master: "../reg.tex"
%%% End: 

\newpage
\addcontentsline{toc}{subsection}{Indications for Use Statement}

\begin{center}
  \huge{Indications for Use Statement}\\[.5in]
\end{center}


\onehalfspacing

510(k) Number (if known): N/A \\
Device Name: eyeScan Mezzo \\
Indications for Use: 

%%% Local Variables: 
%%% mode: latex
%%% TeX-master: "../reg"
%%% End: 

\newpage
\subsection{510(k) Statement}

I certify that, in my capacity as product development scientist, I
will make available all information included in this premarket
notification on safety and effectiveness within 30 days of request by
any person if the device described in the premarket notification
submission is determined to be substantially equivalent. The
information I agree to make available will be a duplicate of the
premarket notification submission, including any adverse safety and
effectiveness information, but excluding all patient identifiers, and
trade secret and confidential commercial information, as defined in 21
CFR 20.61.

\begin{figure}[H]
  \includegraphics[width=0.35\linewidth]{imgs/ja-sig}
\end{figure}

\noindent Joseph Abrahamson \\
\today


%%% Local Variables: 
%%% mode: latex
%%% TeX-master: "../reg"
%%% End: 

\newpage
\subsection{Truthful And Accurate Statement}

I certify that, in my capacity as product development scientist, I
believe to the best of my knowledge, that all data and information
submitted in the premarket notification are truthful and accurate and
that no material fact has been omitted.

\begin{figure}[H]
  \includegraphics[width=0.35\linewidth]{imgs/ja-sig}
\end{figure}

\noindent Joseph Abrahamson \\
\today

%%% Local Variables: 
%%% mode: latex
%%% TeX-master: "../reg"
%%% End: 

\newpage
\addcontentsline{toc}{subsection}{Class I Summary and Certification}
\singlespacing
\begin{center}
  \large{Class I Summary and Certification}
\end{center}

\onehalfspacing

I certify that, in my capacity as product development scientist that I
this device best claims substantial equivalence to a Class I device.

\begin{figure}[H]
  \includegraphics[width=0.35\linewidth]{imgs/ja-sig}
\end{figure}

\noindent Joseph Abrahamson \\
\today

%%% Local Variables: 
%%% mode: latex
%%% TeX-master: "../reg"
%%% End: 
 % Certified SE to Class I instead of CIII
\newpage
\subsection{Financial Certification or Disclosure Statement}

I certify that, in my capacity as product development scientist that
no clinical studies were performed and thus no financial disclosure is
possible or required.

\begin{figure}[H]
  \includegraphics[width=0.35\linewidth]{imgs/ja-sig}
\end{figure}

\noindent Joseph Abrahamson \\
\today

%%% Local Variables: 
%%% mode: latex
%%% TeX-master: "../reg"
%%% End: 

\newpage
\addcontentsline{toc}{subsection}{Declarations of Conformity and Summary Reports}
\singlespacing
\begin{center}
  \large{Declarations of Conformity and Summary Reports}
\end{center}

\onehalfspacing

We chose to not rely on a recognized standard or guidance for any part
of the device design or testing. Instead, we have developed a set of
independent testing protocols. Device testing will be conducted using
this independently generated testing protocol and will meet specified
acceptence criteria before the device is marketed. 



% Real sections
\setcounter{subsection}{0}
\subsection{Executive Summary}
\subsection{Device Description}

The described device (eyeScan Mezzo) is a bench-sized digitizer able to accurately measure the three-dimensional surface structure of small organic forms. Specifically designed to measure the convex shape of dissected rat eyes down to \SI{2}{\micro m} resolution its likely usage would involve research aims with respect to eye shape, morphology, or development especially using genetically modified mouse models.

In order to achieve these requirements the device must utilize an LED micrometer, two motor/encoder acutator systems, and a system controller. The device must be able to accurately position the micrometer and the loaded object to the \SI{2}{\micro m} level.

Additionally, the device uses a sophisticated loading harness to hold and rotate the object without needing to account for object deformation under stress. This harness involves a number of opticaly clear pieces and a methyl cellulose support polymer.

\subsection{Substantial Equivalence Discussion}

\subsection{Proposed Labeling}

The device labeling will include three general categories: packaging,
marketing copy, and operation manuals. Due to the exception of section
502(f) as orchestrated by part 801.125, the operations manual will be
distributed electronically.

\subsubsection{Packaging}
The device is marketed at an operational research lab and therefore
does not require intensive package marketing. Instead an austere
recycled carboard box tied with twine is labeled simply with the
device name (eyeScan Mezzo) and appropriate device IDs.

\subsubsection{Operations Manual}
The operations manual will consist of an easily accessible electronic PDF document divided
into three subsections. The first will detail the process for making
the solution in which the eyeball is suspended; the second will
describe how to suspend the eyeball in solution; and the third will
consist of instructions for connecting the entire device to a
computer.

The suspension solution is polydimethylsiloxane (PDMS), to be made
using the Sylgard 184 Silicone Elastomer Kit included with the device. 1 gram of base solution
and 0.03 grams of curing agent are to be combined in an optical glass
tube at room temperature using a pipette. The resulting solution is to
be well mixed.

To suspend the eyeball in solution, the eyeball will be rinsed
using aqueous water. Inverted tweezers will be used to pick up the
eyeball by the optic nerve and gently suspend it in the center of the
tube containing polymer solution. The tube will then be capped tightly
and placed into the top slot of the scanning machine.

The device’s user interface will consist of two wires. One wire will
plug into a standard United States outlet of 120 V at 60 Hz and the
other will be a USB cable that plugs into a standard USB drive in a
computer. This will allow the device to transmit scanning data to a
computer for three-dimensional reconstruction. The device will also
have a power on/off switch at the back.

The Sylgard 184 Silicone Elastomer kit will include a brief operations
manual along with MSDS information in paper format.

\subsubsection{Marketing Copy}

\paragraph{Description}
The device, named eyeScan Mezzo, is a bench-sized digitizer able to  and intended to accurately measure and create digital meshes corresponding to the three-dimensional surface structure of dissected rat eyes. It is accurate to a 2 um resolution. EyeScan Mezzo is not intended for use in the diagnosis or treatment of diseases for live animals or humans. Device testing is conducted using developer generated testing protocol and will meet specified acceptance criteria before the device is marketed. Full testing of the device's EMC characteristics is performed according to IEC 60601-1-2 Part I.

\paragraph{Safety and Effectiveness Considerations}

EyeScan Mezzo is intended to be operated only by trained laboratory personnel in ophthalmological research settings only. It poses no known immediate health hazard to the operator. 

\paragraph{Contraindications, Warnings, Precautions}

PDMS is toxic to humans, and as such, contact with skin and organs should be avoided. If PDMS does come into contact with the skin, eyes, or body, one should immediately clean off PDMS from body part using a paper towel and rinse contaminated area under running water. Refer to the MSDS for further instruction.

Insertion of objects other than that indicated in the operations manual may damage the device.
If any foreign object enters the device, the operator should turn off the device and cut off the power supply to the device immediately. The operator should then remove the protection casing of the device, remove the foreign object, and reinstall the protection casing. If the device is damaged to the point of malfunction, the operator should contact the manufacturer for repairs. 

A liquid spill inside the machine may damage electronic components and pose an electrical hazard.
If a liquid spill occurs, the operator should turn off the device and cut off the power supply to the device immediately. The operator should then send in a request to the manufacturer to replace damaged parts.

Proper laboratory attire such as latex gloves, laboratory coats, and goggles should be worn before operating the device. 

Injury may result if the device is operated by untrained personnel. 


%%% Local Variables: 
%%% mode: latex
%%% TeX-master: "../reg"
%%% End: 

\newpage
\subsection{Sterilization and Shelf Life}

This device is not sold as sterile. During operation, sterility
doesn't impact the safety of effectiveness of the device.

\newpage
\subsection{Biocompatibility}

This device contains no components which come into direct or indirect
contact with patients. The device is intended for use in research
settings, not clinical settings.

Furthermore, we claim substantial equivalence with microscopes and
accessories (regulation number: 864.3600, product code:
KEH). Microscopes and accessories are identified as optical
instruments used to enlarge images of specimens, preperations, and
cultures for medical purposes. Our device thus makes use of identical
materials used in devices claimed to be substantially equivalent, as
it is intended for use with dissected eyeballs. 


\subsection{Software}

\subsection{Electromagnetic Compatibility and Electrical Safety}

Our device does contain a number of sensitive electronic
devices. Electrical failure of the device cannot result in safety
hazards as under electrical interference the device will simply fail
to deliver results. Additionally, the components of the device are
hardened to electrical shock and resonance. Full testing of the
device's EMC characteristics will be performed according to EC
60601-1-2 Part I.

There is no patient contact with electrical components of the device
as the device is intended to be used exclusively with dissected
eyeballs in research settings.


%%% Local Variables: 
%%% mode: latex
%%% TeX-master: "../reg"
%%% End: 


%\subsection{Performance Testing --- Lab, Animal, Clinical}
% Not required

\section{Discussion}
\label{sec:discussion}

\section{Conclusion}
\label{sec:conclusion}


\newpage
\addcontentsline{toc}{section}{References}
\bibliographystyle{unsrt}
\bibliography{../tex/bibl}

\end{document}
%%% Local Variables: 
%%% mode: latex
%%% TeX-master: t
%%% End: 
