%%% Local Variables: 
%%% mode: latex
%%% TeX-master: t
%%% End: 
\documentclass{article}
\usepackage{../tex/mysty}
\begin{document}

\maketitlepage{Project Update}{Sanjay Challa}

\tableofcontents
\newpage

\section{Project Description}
\label{sec:project-description}

\subsection{Problems in Current Research}
\label{sec:probl-curr-rese}

The current cutting edge of ophthamological research is becoming
increasingly precise, especially driven by the connection between
gross anatomical shape and eye health and function. Particularly, new
research is interested in precision measurement of the exact exterior
shape of the eye [CITE]. While historically the eye shape was
parameterized by three axial lengths, the anterior-posterior (AP)
axis, the nasal-temporal (NT) axis, and the superior-inferior (SI)
axis, new studies are also interested in non-linear, higher-order
parameterizations such as sphericity or even elliptical properties of
surface splines at any location.

An example domain for this research is the investigation of the
genetic factors involved in myopia. Myopia, an extremely common [HOW
COMMON] disesase impacting eye function and focus, is well-known to be
caused by environmental factors, such as sustained near-viewing, but
is increasingly shown to also have a genetic correlate [CITE
2,3,4]. Studies of the interaction of genetic factors with eye shape
seek to determine both ultimate and developmental genetic effects
which result in myopic eyes. Since the symptoms of myopia are directly
caused by overextension of the AP axis leading to a focal point that
resides within the cavity of the eye instead of at the sensitive
retinal well, these researchers are first interested in accurate
measurement of the AP axis [CITE 6], but want to consider more
sophisticated deformations in order to truly understand the genetic
basis [CITE 5].

In order to perform genetic testing, much of this research is carried
out on mouse animal models due to availability, ease of genetic
manipulation, and affordability [CITE 5]. Unfortunately, this means
that the observations are made on the mouse eye which ranges between
1.5 and 4 millimeters in diamter. At this size, affective anatomical
deformations occur at resolutions of 5 microns. Current research
methods include laborious, error-prone, unrepeatable manual micrometry
[CITE 6]; expensive and low-resolution MRI/PET imaging [CITE 7];
error-inducing histological sectioning [CITE 5]; or complex,
inefficient optical interferometry [CITE 5,8].

\subsection{The Opportunity}
\label{sec:opportunity}

Generally, this research is highly impeded by the lack of measurement
techniques that can capture the sophisticated, high-resolution
deformations of the mouse eye that are most interestng while
maintaining repeatability, affordability, and speed.

A lab-bench sized, automated digitizer which can handle small, organic
objects such a dissected mouse eyes would be able to fill this
technical void in the ongoing research and has been called for
repeatedly in published literature [CITE]. While other technical
methods involve complex optical manipulations, the current technique
of manual measurement, especially aided by high-precision micrometry
such as that provided by high-throughput industrial LED micrometers,
could fill all the needs of the researchers if manipulation and
measurement using the micrometer could be automated.

The ideal design advance involves a high-precision articulation frame
which locates and manipulates the measurement plane of an attached
micrometer in order to scan over the full geometry of the eye before
being sent to a computer controller for decoding and construction of a
digital, 3D model of the scanned object. Such a device could quickly,
precisely, and repeatably provide the full gross geometric shape of a
dissected eye and then provide for software to statistically analyze
trends in shape over populations and experimental treatments.

In this case, genetic myopia researchers could correlate genetic
manipulations with exact and non-parametric anatomical changes above
and beyond simple 3-axis measurements.

\subsection{Market Analysis}
\label{sec:market-analysis}

The device suggested would be, by reference, marketable as a
high-preicision general lab device. Typical prices range between
\$30,000 and \$100,000 and so the could be marketed within that
range. Additionally, this device is targeted at a niche with
relatively little current competition since researchers suffice using
the expensive, inaccurate, or difficult current technology.

The proximate market includes genetic myopia researchers and, slightly
more generally, all researchers involved in advanced ophthalmological
research involving gross anatomical measurements. Approximately 500
researchers working at 70 medical research institutions around the US
would benefit from the device suggesting between 50 and 120 initial
purchases and a market of \$2 to \$10 million.

An extended market exists, however, in all research involved in
anatomical mouse models or small-structure human models. For instance,
ear research is highly concerned with the exact shape and function of
the ossicles. This research is likely carried out at a similar number
of research institutions thus increasing the market to two or three
purchases per institution and raising the market value to \$30 million
or higher.

\subsection{Potential FDA Regulation}
\label{sec:potent-fda}

This device is not considered an FDA regulated medical device. It has
no direct or indirect interaction with living tissue, human or
animal. It is not used for diagnosis, cure, mitigation, treatment, or
prevention of disease. Moreover, the device is not even intended for
sterile use. The device is instead intended strictly for research
purposes.


\newpage
\bibliographystyle{unsrt}
\bibliography{../tex/bibl}

\end{document}
