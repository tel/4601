%%% Local Variables: 
%%% mode: latex
%%% TeX-master: t
%%% End: 
\documentclass{article}
\usepackage{../tex/mysty}
\begin{document}
 
\maketitlepage{Societal Impact Report}{Ginger Tsai}
 
\setcounter{tocdepth}{2}
\tableofcontents
\newpage
\listoftables
\listoffigures
\newpage
 
 
\section*{Executive Summary}
\label{sec:exec-summary}
 
 
We have considered and analyzed five possible impacts our device may have on society. 
 
\newpage
 
 
\section{Introduction}
\label{sec:Introduction}
 
The cutting edge of ophthalmological research is focused on extremely precise measurements of the eye, as research suggests a strong connection between gross anatomical shape and eye health and function. Particularly, new research is interested in precision measurement of the exact exterior shape of the eye\cite{atchison04,zhou99:genes,zhou99:models,guggenheim04,wallman04}. While historically eye shape was parametrized by three axial lengths, the anterior-posterior (AP) axis, the nasal-temporal (NT) axis, and the superior-inferior (SI) axis, new studies are interested in non-linear, higher-order parameterizations involving sphericity or even elliptical properties of surface splines at any location on the eye.
 
An example domain for this research is the investigation of the genetic factors involved in myopia, which is thought to be a symptom of greater changes to overall eye shape \cite{atchison04}. Myopia, which has a 25\% prevalence in western cultures and nearly 80\% prevalence in some Asian populations\cite{rajan98}, is an extremely common disease impacting eye function and focus. It is well-known to be caused by environmental factors such as sustained near-viewing, but is also increasingly shown to also have a genetic correlate\cite{zhou99:genes,zhou99:models,schmucker04}. Studies of the interaction of genetic factors with eye shape seek to determine both ultimate and developmental genetic effects which result in myopic eyes.   Since the symptoms of myopia are directly caused by over-extension of the AP axis, leading to a focal point that resides within the cavity of the eye instead of at the sensitive retinal well, these researchers are first interested in accurate measurement of the AP axis\cite{wallman04}, but want to consider more sophisticated deformations as in the NT and SI axes to truly understand the effect of genetics on the development of a myopic eye\cite{schaeffel04} and thus of gross eye shape \cite{atchison04}.  
 
In order to perform such genetic analysis, much of this research is carried out on mouse animal models due to availability, ease of genetic manipulation, and affordability\cite{schaeffel04}. Unfortunately, this means that the observations are made on the mouse eye, which ranges between 1.5 to 4 millimeters in diameter. At this size, affective anatomical deformations occur at resolutions of 5 microns. Current research methods include laborious, error-prone, unrepeatable manual micrometry\cite{wallman04}; expensive and low-resolution MRI/PET imaging\cite{atchison04}; error-inducing histological sectioning\cite{schaeffel04}; or complex, inefficient optical interferometry\cite{guggenheim04,schaeffel04}. A device capable of three-dimensional imaging with high precision is thus needed to facilitate progress of such ophthalmological research.
 
To meet these needs, the team has proposed a device which implements complete three-dimensional reconstruction by scanning over the eye with a LED micrometer. The device, pictured in figure \ref{fig:schematic}, consists roughly of three parts including an eye-actuating harness, the micrometer frame, and the computer interface. The eye harness is the primary physical interaction point for the user and thus enables easy replacement of specimens into the device. Its purpose is to hold the eye and fix it rigidly for acceleration without impacting the optical path for the micrometer. To this end, it consists of a pair of reverse-action tweezers which are fitted into a cylindrical cuvette filled with a high-viscosity methyl cellulose buffer suspension. When the cuvette is inserted into the frame of the device, it is placed against a corrective lens which corrects for the optical distortion from the cylinder. The harness then attaches to a motor supplemented by an angular encoder, which can then accelerate the tube while keeping high-resolution data on its current theta-position.
 
The frame itself holds the motors and the micrometer heads (receiver and transmitter). The heads are attached to a microscope frame augmented with another motor/encoder pair to provide linear articulation in the z-direction while maintaining high-resolution position data. Finally, the data from the two encoders and the micrometer is fed into a National Instruments Single-board Real-time Input Output device (sbRIO) which provides real-time feedback and control while collecting data for online reconstruction. The reconstruction algorithm itself is a modified back-projection algorithm supplemented by cosine interpolation in the z-direction.
 
The entire device operates by accelerating the eye harness up to a fixed angular velocity and then taking measurements while adjusting the micrometer head location. This results in a dataset which has very high resolution data at locations on the eye with small circumferences, but the velocities of both actuators can be chosen to limit the minimum resolution along the widest circumferences as to be within spec. While the over-scanning that occurs in low circumference areas is unnecessary and increases scan time, the methyl-cellulose buffer increases eye survival time to several hours, thus relaxing the need for rapid scanning.
 
In the process of designing and manufacturing a device, it is important to consider the various impacts the device may have on society throughout its lifetime. The team has analyzed five possible impacts of the proposed device: demographics and economic impact, medical ethics, public policy, and environmental concerns. 
 
 
\subsection{Demographics and Economic Impact}
\label{sec:Demographics}
 
The target users of the device will be researchers in ophthalmology, particularly in
fields requiring accurate measurements of an eye. An estimated number
of 500 researchers working at 70 medical research institutions would
benefit from the device \cite{Nickerson}. The device will introduce new avenues to
study the eye while helping researchers conduct their research better
and faster. Specifically, the device enables three-dimensional
analysis allowing the researchers to study more variables, such as
sphericity of the eyeball, in quantitative, less strenuous, and less
time-consuming ways. The improved performance that the device would
allow would translate into savings, in terms of both time and money,
for both researchers and organizations such as the NIH who fund the
researches. This in turn would allow money from the general public and
resources at research instutions to be invested in making discoveries
on more fronts, advancing scientific knowledge at a greater
rate. 
 
Another demographic on which the device would have an impact includes
the niche market of those who specialize in making optically precise
glassware. Two key components in the device are the optically precise
cuvette and corrective lens, both which would be purchased from
specific manufacturers. Additionally, the device could also
potentially aid mechanical engineers studying ball bearings and
archaeologists studying small bone fossils or bone fragments. These
populaces would  benefit from use of the device in terms of saving time and money. 

\subsection{Environmental Impact (Joe)}
\label{sec:environment}
 
The device, intended for extended use in a controlled setting, is
posed to be highly environmentally conscientious. Due to the
relatively small number of potential buyers, the manufacture,
packaging, and shipping can all be optimized to minimize environmental
impact. The device itself produces little to no waste during operation
and presents nominal energy draw. Finally, should the device be
disposed of by the end user, it would be easy to reclaim the
highly-modular components for reuse either in new devices or in other
projects.
 
Our proposed manufacturing methods need to serve a relatively small
consumer base of approximately 500 research labs in the US. At this volume,
it is likely that our devices will be custom fabricated using CNC
milling instead of more volume-dependent manufacturing methods such as
casting. The environmental impact here is that the manufacturing
process can operate with a smaller footprint while outputting a
product that is more likely to last through more cycles.
 
In order to continue with the green methods already proposed, our
device can be packaged using attractive yet cheap recycled
materials. Moreover, so long as the device can be user-tested for
designs which are easy and intuitive to use, we can eschew providing a
printed manual and alternatively provide an online manual, saving
further packaging and resource consumptionn.
 
During its lifetime of use, the device consumes very little
material. All pieces are high quality, robust, and often optically
precise requiring that each component is cleaned and reused instead of
being disposed of. While it is possible that parts such as the
mounting cuvet could shatter if mishandled, care should be taken on
the part of the consumer in reducing the frequency of this sort of
waste, especially since optically-precise cuvets are expensive.
 
Finally, since the component parts are robust and modular, if a device
does fail it is likely that it can either be easily repaired by a
specialist or recycled to reuse the component parts which are still
operational. Here, the modularity enables us to minimize the impact of
the disposal of damaged, failed devices.
 
The proposed device therefore is well poised to cause minimal
environmental damage through its manufacture, sale, operation, and
disposal. We support this capability with a commitment to the
environmentally friendly processes available to us by erring on the
side of quality and reusability.
 
 
\subsection{Medical Ethics (Joe)}
\label{sec:blah-blah-blah}
 
As with any research-oriented goal, the distant effects of proximal
successes are difficult to gauge; however, since this device most
immediately affects our knowledge of the genetic factors in eye
development --- especially abnormal, myopic eye development --- we can
expect some possibility of discovery there. With any advance in
genetics, there is concern for treatment, identification, and
policy-level decisions concerning those affected by any genetic
predispositions.
 
For instance, it is foreseeable that the United States Air Force, notorious for
rejecting on valid grounds those without 20/20 vision, could extend
their recruitment programs to reject those who have a high chance of
developing myopia (or demanding that they receive elective surgeries
to correct for it). In this case, it is likely a justifiable threat,
but more generally these possibilities can be concerning. The proposed
device could be key in discovering and proving these genetic tests,
and therefore it is important that we consider our part in the
possible development of these policies.
 
Fortunately, the principal disease being studied is myopia. In the
vast majority of cases there is already a non-invasive, popular
treatment: common eyeglasses. Additionally, as time goes forward it is
likely that our surgical laser-corrective methods will both improve
and drop in price making that an attractive option. Finally, genetic
causes of eye deformation may even be affected \textit{post hoc} by
treating the secondary steps in the development of these diseases. All
of these methods could gain both in awareness, affordability, and
availability from the research enabled by our device.
 
Even in the worst foreseeable case, should policy fail to prevent
widespread unethical abuse of genetic information, enabling people to
more accurately predict myopia is unlikely to cause much harm. Once
again, there are ample treatment and corrective options available.
 
To this end, we believe our device to be well in-line with the ethical
principles of today. Its deployment in medical contexts is unlikely
to cause harm to people and instead may provide new opportunities to
reclaim ability that normally would be lost to ocular deformation.
 
 
 
\subsection{Shuyen's Public Policy}
\label{sec:Public Policy}
 
The design solution will not be regulated by the Food and Drug Administration. It is not intended
to diagnose, treat, mitigate, or cure any disease, nor will it come into
contact with live tissue in any way; hence, it is not considered a medical device as defined in the Federal Food, Drug,
and Cosmetic Act (FFDCA) section 201(h) \cite{fdaguidelines}. Also, since the device will not be used
on live tissues, there is little regulation necessary concerning the sterility
or the material properties of the device beyond possible biological contamination from the dissected eyeball, whose treatment and disposal is regulated by
the local regulations of the research facility.  Even though the LED micrometer of the device employs visible light, it is minimally
regulated by the FFDCA, as it can be classified as a Class I
laser product according to sections 1040.10, for it "does not
permit access during the operation to levels of laser radiation in
excess of the accessible emission limits" \cite{fdaguidelines}. 
 
 
 
% TODO: Ginger's section
 
\newpage
\addcontentsline{toc}{section}{References}
\bibliographystyle{unsrt}
\bibliography{../tex/bibl}
 
\end{document}
 
