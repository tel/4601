%%% Local Variables: 
%%% mode: latex
%%% TeX-master: t
%%% End: 
\documentclass{article}
\usepackage{../tex/mysty}
\begin{document}

\maketitlepage{Societal Impact Report}{Ginger Tsai}

\setcounter{tocdepth}{2}
\tableofcontents
\newpage
\listoftables
\listoffigures
\newpage


\section*{Executive Summary}
\label{sec:exec-summary}

% TODO: Ginger's section


\newpage


\section{Introduction}
\label{sec:introduction}

% TODO: Ginger's section


\section{Notable Concerns}
\label{sec:concerns}

% Introductory sentence (Ginger)


\subsection{Environmental Impact (Joe)}
\label{sec:environment}

The device, intended for extended use in a controlled setting, is
posed to be highly environmentally conscientious. Due to the
relatively small number of potential buyers, the manufacture,
packaging, and shipping can all be optimized to minimize environmental
impact. The device itself produces little to no waste during operation
and presents nominal energy draw. Finally, should the device be
disposed of by the end user, it would be easy to reclaim the
highly-modular components for reuse either in new devices or in other
projects.

Our proposed manufacturing methods need to serve a relatively small
consumer base of perhaps 300 research labs in the US. At this volume,
it is liable that our devices will be custom fabricated using CNC
milling instead of more volume-dependent manufacturing methods such as
casting. The environmental impact here is that the manufacturing
process can operate with a smaller footprint while outputting a
product that is more likely to last through more cycles.

In order to continue with the green methods already proposed, our
device can be packaged using attractive yet cheap recycled
materials. Moreover, so long as the device can be user-tested for
designs which are easy and intuitive to use, we can eschew providing a
printed manual and alternatively provide an online manual, saving
further packaging and resource consumptionn.

During its lifetime of use, the device consumes very little
material. All pieces are high quality, robust, and often optically
precise requiring that each component is cleaned and reused instead of
being disposed of. While it is possible that parts such as the
mounting cuvet could shatter if mishandled, care should be taken on
the part of the consumer in reducing the frequency of this sort of
waste, especially since optically-precise cuvets are expensive.

Finally, since the component parts are robust and modular, if a device
does fail it is likely that it can either be easily repaired by a
specialist or recycled to reuse the component parts which are still
operational. Here, the modularity enables us to minimize the impact of
the disposal of damaged, failed devices.

The proposed device therefore is well poised to cause minimal
environmental damage through its manufacture, sale, operation, and
disposal. We support this capability with a commitment to the
environmentally friendly processes available to us by erring on the
side of quality and reusability.


\subsection{Medical Ethics (Joe)}
\label{sec:blah-blah-blah}

As with any research-oriented goal, the distant effects of proximal
successes are difficult to gage; however, since this device most
immediately affects our knowledge of the genetic factors in eye
development --- especially abnormal, myopic eye development --- we can
expect some possibility of discovery there. With any advance in
genetics, there is concern for treatment, identification, and
policy-level decisions concerning those affected by any genetic
predispositions.

For instance, it's foreseeable that the US Air Force, notorious for
rejecting on valid grounds those without 20/20 vision, could extend
their recruitment programs to reject those who have a high chance of
developing myopia (or demanding that they receive elective surgeries
to correct for it). In this case, it's likely a justifiable threat,
but more generally these possibilities can be concerning. The proposed
device could be key in discovering and proving these genetic tests,
and therefore it is important that we consider our part in the
possible development of these policies.

Fortunately, the principal disease being studied is myopia. In the
vast majority of cases there is already a non-invasive, popular
treatment: common eyeglasses. Additionally, as time goes forward it is
likely that our surgical laser-corrective methods will both improve
and drop in price making that an attractive option. Finally, genetic
causes of eye deformation may even be affected \textit{post hoc} by
treating the secondary steps in the development of these diseases. All
of these methods could gain both in awareness, affordability, and
availability from the research enabled by our device.

Even in the worst foreseeable case, should policy fail to prevent
widespread unethical abuse of genetic information, enabling people to
more accurately predict myopia is unlikely to cause much harm. Once
again, there are ample treatment and corrective options available.

To this end, we believe our device to be well in-line with the ethical
principles of today. Its deployment in medical contexts is unlikely
to cause harm to people and instead may provide new opportunities to
reclaim ability that normally would be lost to ocular deformation.

\subsection{Shuyen's Demographic}
\label{sec:Demographic}

The design solution is a high resolution three-dimensional imaging
device.  Its target users will be researchers in ophthalmology,
particularly in the fields requiring accurate measurements of an
eye. Given the task, field of study, and the environment of operation,
the target users will need the device to: measures the mouse model
eyes with high accuracy; have user-friendly interfaces; be mostly
automated; and have similar size and weight of a medium laboratory
instrument (such as a counter-top centrifuge). An estimated number of
500 researchers working at 70 medical research institutions would
benefit from the device. The device will introduce new avenues to
study the eye and help the researchers to conduct their research
better and faster. The device enables three-dimensional analysis
allowing the researchers to study more variables such as sphericity of
the eyeball quantitatively in a less strenuous, less time-consuming
way. The enhanced performance of researchers implies savings for the
people and the organizations, such as NIH or R\&D departments, who
fund the researches because the researchers can perform more
efficiently with the same amount of funding. Mechanical engineers
studying ball bearing and archaeologists studying bone fossil pieces
are also potential users who might be interested in the device.
	
\subsection{Shuyen's Public Policy}
\label{sec:Public Policy}

The design solution will not be regulated by the FDA. The device is
not considered a medical device as defined in the Federal Food, Drug,
and Cosmetic Act (FFDCA) section 201 (h). The device is not intended
to diagnose, treat, mitigate, or cure any disease; or come into
contact with live tissue in any way. Since the device will not be used
on live tissues, these is little regulation concerning the sterility
or the material properties of the device. The one concern is the
possible biological contamination from the dissected eyeball, in which
case the treatment and disposal of the contamination is regulated by
the local regulations of the research facility.  Even thought a
component of the device employs visible light, it is minimally
regulated by the FFDCA as the component can be classified as Class I
laser product according to sections 1040.10 because it "does not
permit access during the operation to levels of laser radiation in
excess of the accessible emission limits."



% TODO: Ginger's section

\newpage
\addcontentsline{toc}{section}{References}
\bibliographystyle{unsrt}
\bibliography{../tex/bibl}

\end{document}
