\documentclass{article}
\usepackage{../tex/mysty}
\begin{document}


\maketitlepage{Final Report}{Shuyen Liu}

\setcounter{tocdepth}{3}
\tableofcontents
\newpage

\section*{Executive summary}
\label{sec:exec-summary}

\newpage

\section{Project Description}
\label{sec:project-description}

Myopia is one of the most common ocular diseases in humans, approximately 25% in most 
Western populations and reaches more than 80% in some Asian populations.\cite{rajan98}1 
Current research suggests that there is a strong genetic influence upon the development of the
condition in addition to extrinsic factors.\cite{zhou99:genes,zhou99:models,schmucker04}2,3,4
As a result of such studies, many ophthalmology researchers have turned to
focus on how specific genes affect the development and maintenance of the eye. Mouse
has been the common animal model due to availability, ease of genetic manipulation, and affordability. \cite{schaeffel04}ref 8 
However, the small size of the mouse eye is a critical impediment to research in 
this field, since precise measurements of the size of the eye are essential in determining
the effect of genes and environmental factors on eye growth and disease.\cite{schaeffel04}ref5 
Using the mouse model, researchers need to be able to accurately measure
the Anterior-Posterior axis of the small mouse eyeball, whose effective anatomical deformations 
occur at resolutions of 5 microns. 
Current techniques for characterizing external eye shape (Table 1 of 4600 final) include manual contact
(physical micrometry),\cite{wallman04} ref6 histological sectioning,\cite{schaeffel04}ref5 MRI/CT/PET imaging,\cite{atchison04}ref7 and optical
interferometry.\cite{schaeffel04,guggenheim04}ref5,8 Each technique measures the eye but is either too difficult for common
use or too inaccurate for precise use. Presently, there is no method that can rapidly 
easily and accurately determine the gross exterior shape of the eye at a resolution under 5 μm 
in the lab setting.

A lab-bench sized, automated digitizer which can handle small, organic objects such a dissected mouse
eyes would be able to fill this technical void in the ongoing research and has been called for repeatedly in
published literature. \cite{schaeffel04,atchison04,zhou99:genes,zhou99:models} ref8,1,2,3 of project update
While other technical methods involve complex optical manipulations, the current
technique of manual measurement, especially aided by high-precision micrometry such as that provided by
high-throughput industrial LED micrometers, could fill all the needs of the researchers if manipulation and
measurement using the micrometer could be automated.


The ideal design advance involves a high-precision articulation frame which locates and manipulates the
measurement plane of an attached micrometer in order to scan over the full geometry of the eye before being
sent to a computer controller for decoding and construction of a digital, 3D model of the scanned object.
Such a device could quickly, precisely, and repeatably provide the full gross geometric shape of a dissected
eye and then provide for software to statistically analyze trends in shape over populations and experimental
treatments.
To meet these needs, the team has proposed a device which implements complete three-dimensional
reconstruction by scanning over the eye with a laser-emitting diode (LED) micrometer. The device, pictured
in figure 1 (SIA), consists roughly of three parts including an eye-actuating harness, the micrometer frame, and
the computer interface. The eye harness holds the eye and fix it rigidly for acceleration without impacting the
optical path for the micrometer. It consists of a pair of reverse-action tweezers which are fitted into
a cylindrical cuvette filled with a high-viscosity methyl cellulose buffer suspension. When the cuvette is 
inserted into the frame of the device, it is placed against a corrective lens which corrects for the optical 
distortion from the cylinder (not shown). The harness then attaches to a
motor supplemented by an angular encoder, which can then accelerate the tube while keeping high-resolution
data on its current theta-position.
The frame itself holds the motors and the micrometer heads, which consist of the receiver and transmit-
ter. The heads are attached to a syringe pump frame augmented with another motor/encoder pair to provide
linear articulation in the z-direction while maintaining high-resolution position data. Finally, the data from
the two encoders and the micrometer is fed into a National Instruments Single-board Reconfigurable Input
Output device (sbRIO), which provides real-time feedback and control while collecting data for online re-
construction. The reconstruction algorithm itself is a modified back-projection algorithm supplemented by
cosine interpolation in the z-direction.


The target users of the device will be researchers in ophthalmology, particularly those who conduct studies
requiring precise measurements of an eye. An estimated number of 500 researchers working at 70 medical
research institutions will benefit from the device \cite{Nickerson}ref9 (SIA), which will introduce new avenues to study the eye while
enabling researchers to conduct current research easier and faster. The improved performance supplied by the
device would translate into savings, in terms of both time and money, for both researchers and organizations
that fund them, such as the National Institutes of Health (NIH). This in turn would allow money from the
general public and resources at research institutions to be invested in making discoveries on more fronts,
advancing scientific knowledge at a greater rate.

\section{Engineering Design Specifications}
\label{sec:engin-design-spec}

The most significant aspect that the device needs to have is the ability to take automated
measurements of ocular dimensions at a resolution of 0.5 diopters, which is equivalent to 2
μm for mouse eyes. Upon connection to a computer, the device also needs to be able to 
convert measurements into a 3D interactive reconstruction of the eyeball. Also, the device 
needs to be able to fit onto a typical laboratory bench, into a space of 1mx1mx1m, but it 
does not need to maintain a sterile field.
It is also desirable that the device has high mobility; that is, it can be easily
lifted and carried from one lab bench to another by an average healthy researcher from age
20~50. Therefore, the device should be lightweight; the weight of the device should not
exceed 40 lbs (18 kg).\cite{gross03} ref16 The device should also not be cumbersome; the design of the 
device should be simple and manageable with few external components.
The customer also wants the device to have a simple user interface with which
measurements taken by the device may be transferred onto a connected computer with minimal operation.
The device should also be reusable for up to 4 years given constant
daily use of the device at least 5 times a day, with 1 hour per use.\cite{keyence01}ref17 The device should not
require any maintenance unless one of the components stops functioning.\cite{keyence01}ref17 In addition, the
device should cause little to no vibration during operation such that the accuracy of
measurements is unaffected. Currently, the amount of vibration an eyeball can experience
without affecting measurement accuracy is untested.
Finally, the desires of the customer are that the device be aesthetically pleasing. The device
should also not make very much noise during performance. The noise produced by the
device should not exceed the level of 80 dB, the loudness of the average car.\cite{truax09}ref18
Engineering Characteristics
The device needs to be robust enough to withstand the stresses of everyday lab use.
Robustness is defined as the ability to withstand rough accidental movement such as being
dropped from the top of a lab bench, typically at a height of 3 feet, or shoved sideways on
the bench surface. The device should also be able to hold the weight of small objects no
heavier than 5 pounds, such as lab notebooks and pens. The device needs to be able to
function in a wide range of temperature typically found in the laboratory, typically 22±5 oC,
and as specified earlier, it needs to fit comfortably onto a lab bench within a space of
1mx1mx1m. The non-electrical components of the device should be inert and should not be
affected by small quantities (less than 1 mL) of minimally hazardous solutions such as PBS,
\cite{users_manual}13
which is typically used to keep the eye moist. Electrically, the device needs to function with
120V 60Hz AC, the standard power supply of the United States.19


\section{Prototype Development Discussion}
\label{sec:prot-devel-disc}
[intro]
	The central component of the device is the Keyence LS-7030 LED digital micrometer provided by 
the client. The LED micrometer assumes that light travels between the emitter and receiver in an 
undistorted fashion, and that only the measured object occludes some of the light incident upon the 
receiver, as reviewed in US patent No.6,947,152, High Speed Laser Micrometer. 
	
	A prior art and literature research found that among the current techniques, x-ray computed 
tomography is most relevant to the design solution of the current opportunity because its core measuring 
component is similar to the LS-7030 micrometer in concept. The principle steps in computed tomography 
include measuring attenuation at different angles, reconstructing a slice via back-projection, and repeat the 
step for next slice. The design solution will be based on this process, which requires angular (theta) and 
translational (z) motion and image reconstruction. 

[components]
	The design solution prototype employs two motor for angular and translational motion. The initial 
design focuses on resolution of the resulting image and intended to move the micrometer set both 
translationally and angularly. However, there is a significant moment of inertia due to the weight of the 
micrometer. In addition, the client raised concerns about damaging the core piece (the micrometers). 
Based on these reasons, the solution changed to rotate the target of interest (mouse eyeball), and move 
the micrometer translationally. Based on the tasks needed to be performed as described above and in the 
EDS, the solution requires: core piece that holds mouse eyeball and keeps its shape; motors; connections 
between motor and the core piece and between motor and the micrometer; structural frame work to ensure 
alignment between components; data out put to comupter; and image reconstruction algorithm and user 
interface. 

[frame]
	The device is set up so that the axis of rotation of the theta motor, the eyeball, and the structures 
that support the eyeball is along the same axis. The micrometer heads are attached to their base, 
connecting to a block whose position is controled by the z-motor through a threaded shaft, in such a way 
that the plane of measurement is perpendicular to the axis of rotation. To achieve good alignment, the 
solution uses specially designed pieces to hold components in place. The solution has a base where the 
mechanical components sit on. A frame work, consisting of a top, a bottom, and two legs that connect and 
support the top piece to the bottom piece, attaches to the base with bolts and nuts. The theta-motor sits 
press-fitted in a designed cavity and defines the axis of rotation. The top piece has a hole that is concentric 
to the axis of rotation and serves to maintain alignment of a set of core pieces that holds the eyeball. 

[harness and fixate]
	The core pieces consist of a clear tubing, top and bottom caps, holder on the top cap that holds 
the eyeball in the tubing, and a connector between bottom cap and theta-motor. All of the core pieces are 
cylindrical and have their centers on the axis of rotation. The bottom cap is detachable from the connector 
to theta-motor so the tubing can leave the device to be loaded with new eyeball. The tubing will be loaded 
with eyeball and tissue fixate that keeps the shape of the eyeball. Out side of the core, there will be optical 
components that corrects the path of light that's distorted by the tube and fixate solution.

[micrometers]
	On the opposite side of the two support legs, the guiding and driving shafts and z-motor for 
translational motion sits in simple aluminum housing fixed to the base. A connector block mounted on the 
shafts connects to the base of the micrometer so that the z-motor can drive the micrometer to different z-
position. A computer connects to the motors to control them and reads data from the micrometer for image 
reconstruction. 
	
[optics somewhere...]
[image reconstruction algorithm]

[prototype construction]
	Construction of the prototype employs Zprinter® 310 (inkjet) Printer with special starch powder 
and the Stratasys(Registered) Dimension 1200 FDM (Fused deposition modeling) Printer with ABS 
(acrylonitrile butadiene styrene) material to print the structural framework for review and testing.

[Iteration]
	The first draft of the prototype structural framework came from the Zprinter and showed that the 
structures are sturdy enough to not vibrate much. However, since he Zprinter does not have good enough 
resolution for tight connection and precise position, the FDM Printer is chosen to print the structures to be 
used in final prototype. The second draft used the same blueprint but was printed with the FDM printer. The 
second draft is lighter and has better precision and fitting between parts. It is also strong and does not 
vibrate much.  Third draft is being printed now........
	The first z-motor was was 0.6A mounted on a modified microscope stage. It did not have enough 
power to drive the micrometer and was switched out with a 2A stepper motor on syringe pump.


\section{Performance Testing and Results}
\label{sec:perf-test-results}

\section{Final Design Solution}
\label{sec:final-design-solut}

\section{Discussion}
\label{sec:discussion}

\section{Conclusion and Recommendations}
\label{sec:concl-recomm}

\newpage
\addcontentsline{toc}{section}{References}
\bibliographystyle{unsrt}
\bibliography{../tex/bibl}

\end{document}
%%% Local Variables: 
%%% mode: latex
%%% TeX-master: t
%%% End: 
