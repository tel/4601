\documentclass{article}
\usepackage{../tex/mysty}
\begin{document}


\maketitlepage{Final Report}{Shuyen Liu}

\setcounter{tocdepth}{3}
\tableofcontents
\newpage

\section*{Executive summary}
\label{sec:exec-summary}

\newpage

\section{Project Description}
\label{sec:project-description}

Myopia is one of the most common ocular diseases in humans, approximately 25% in most 
Western populations and reaches more than 80% in some Asian populations.\cite{rajan98}1 
Current research suggests that there is a strong genetic influence upon the development of the
condition in addition to extrinsic factors.\cite{zhou99:genes,zhou99:models,schmucker04}2,3,4
As a result of such studies, many ophthalmology researchers have turned to
focus on how specific genes affect the development and maintenance of the eye. Mouse
has been the common animal model due to availability, ease of genetic manipulation, and affordability. \cite{schaeffel04}ref 8 
However, the small size of the mouse eye is a critical impediment to research in 
this field, since precise measurements of the size of the eye are essential in determining
the effect of genes and environmental factors on eye growth and disease.\cite{schaeffel04}ref5 
Using the mouse model, researchers need to be able to accurately measure
the Anterior-Posterior axis of the small mouse eyeball, whose effective anatomical deformations 
occur at resolutions of 5 microns. 
Current techniques for characterizing external eye shape (Table 1 of 4600 final) include manual contact
(physical micrometry),\cite{wallman04} ref6 histological sectioning,\cite{schaeffel04}ref5 MRI/CT/PET imaging,\cite{atchison04}ref7 and optical
interferometry.\cite{schaeffel04,guggenheim04}ref5,8 Each technique measures the eye but is either too difficult for common
use or too inaccurate for precise use. Presently, there is no method that can rapidly 
easily and accurately determine the gross exterior shape of the eye at a resolution under 5 μm 
in the lab setting.

A lab-bench sized, automated digitizer which can handle small, organic objects such a dissected mouse
eyes would be able to fill this technical void in the ongoing research and has been called for repeatedly in
published literature. \cite{schaeffel04,atchison04,zhou99:genes,zhou99:models} ref8,1,2,3 of project update
While other technical methods involve complex optical manipulations, the current
technique of manual measurement, especially aided by high-precision micrometry such as that provided by
high-throughput industrial LED micrometers, could fill all the needs of the researchers if manipulation and
measurement using the micrometer could be automated.


The ideal design advance involves a high-precision articulation frame which locates and manipulates the
measurement plane of an attached micrometer in order to scan over the full geometry of the eye before being
sent to a computer controller for decoding and construction of a digital, 3D model of the scanned object.
Such a device could quickly, precisely, and repeatably provide the full gross geometric shape of a dissected
eye and then provide for software to statistically analyze trends in shape over populations and experimental
treatments.
To meet these needs, the team has proposed a device which implements complete three-dimensional
reconstruction by scanning over the eye with a laser-emitting diode (LED) micrometer. The device, pictured
in figure 1 (SIA), consists roughly of three parts including an eye-actuating harness, the micrometer frame, and
the computer interface. The eye harness holds the eye and fix it rigidly for acceleration without impacting the
optical path for the micrometer. It consists of a pair of reverse-action tweezers which are fitted into
a cylindrical cuvette filled with a high-viscosity methyl cellulose buffer suspension. When the cuvette is 
inserted into the frame of the device, it is placed against a corrective lens which corrects for the optical 
distortion from the cylinder (not shown). The harness then attaches to a
motor supplemented by an angular encoder, which can then accelerate the tube while keeping high-resolution
data on its current theta-position.
The frame itself holds the motors and the micrometer heads, which consist of the receiver and transmit-
ter. The heads are attached to a syringe pump frame augmented with another motor/encoder pair to provide
linear articulation in the z-direction while maintaining high-resolution position data. Finally, the data from
the two encoders and the micrometer is fed into a National Instruments Single-board Reconfigurable Input
Output device (sbRIO), which provides real-time feedback and control while collecting data for online re-
construction. The reconstruction algorithm itself is a modified back-projection algorithm supplemented by
cosine interpolation in the z-direction.


The target users of the device will be researchers in ophthalmology, particularly those who conduct studies
requiring precise measurements of an eye. An estimated number of 500 researchers working at 70 medical
research institutions will benefit from the device \cite{Nickerson}ref9 (SIA), which will introduce new avenues to study the eye while
enabling researchers to conduct current research easier and faster. The improved performance supplied by the
device would translate into savings, in terms of both time and money, for both researchers and organizations
that fund them, such as the National Institutes of Health (NIH). This in turn would allow money from the
general public and resources at research institutions to be invested in making discoveries on more fronts,
advancing scientific knowledge at a greater rate.

\section{Engineering Design Specifications}
\label{sec:engin-design-spec}

The most significant aspect that the device needs to have is the ability to take automated
measurements of ocular dimensions at a resolution of 0.5 diopters, which is equivalent to 2
μm for mouse eyes. Upon connection to a computer, the device also needs to be able to 
convert measurements into a 3D interactive reconstruction of the eyeball. Also, the device 
needs to be able to fit onto a typical laboratory bench, into a space of 1mx1mx1m, but it 
does not need to maintain a sterile field.
It is also desirable that the device has high mobility; that is, it can be easily
lifted and carried from one lab bench to another by an average healthy researcher from age
20~50. Therefore, the device should be lightweight; the weight of the device should not
exceed 40 lbs (18 kg).\cite{gross03} ref16 The device should also not be cumbersome; the design of the 
device should be simple and manageable with few external components.
The customer also wants the device to have a simple user interface with which
measurements taken by the device may be transferred onto a connected computer with minimal operation.
The device should also be reusable for up to 4 years given constant
daily use of the device at least 5 times a day, with 1 hour per use.\cite{keyence01}ref17 The device should not
require any maintenance unless one of the components stops functioning.\cite{keyence01}ref17 In addition, the
device should cause little to no vibration during operation such that the accuracy of
measurements is unaffected. Currently, the amount of vibration an eyeball can experience
without affecting measurement accuracy is untested.
Finally, the desires of the customer are that the device be aesthetically pleasing. The device
should also not make very much noise during performance. The noise produced by the
device should not exceed the level of 80 dB, the loudness of the average car.\cite{truax09}ref18
Engineering Characteristics
The device needs to be robust enough to withstand the stresses of everyday lab use.
Robustness is defined as the ability to withstand rough accidental movement such as being
dropped from the top of a lab bench, typically at a height of 3 feet, or shoved sideways on
the bench surface. The device should also be able to hold the weight of small objects no
heavier than 5 pounds, such as lab notebooks and pens. The device needs to be able to
function in a wide range of temperature typically found in the laboratory, typically 22±5 oC,
and as specified earlier, it needs to fit comfortably onto a lab bench within a space of
1mx1mx1m. The non-electrical components of the device should be inert and should not be
affected by small quantities (less than 1 mL) of minimally hazardous solutions such as PBS,
\cite{users_manual}13
which is typically used to keep the eye moist. Electrically, the device needs to function with
120V 60Hz AC, the standard power supply of the United States.19
\section{Prototype Development Discussion}
\label{sec:prot-devel-disc}

\section{Performance Testing and Results}
\label{sec:perf-test-results}

\section{Final Design Solution}
\label{sec:final-design-solut}

\section{Discussion}
\label{sec:discussion}

\section{Conclusion and Recommendations}
\label{sec:concl-recomm}

\newpage
\addcontentsline{toc}{section}{References}
\bibliographystyle{unsrt}
\bibliography{../tex/bibl}

\end{document}
%%% Local Variables: 
%%% mode: latex
%%% TeX-master: t
%%% End: 
