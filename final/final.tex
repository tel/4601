\documentclass{article}
\usepackage{../tex/mysty}
\begin{document}


\maketitlepage{Final Report}{Shuyen Liu}

\setcounter{tocdepth}{3}
\tableofcontents
\newpage

\section*{Executive summary}
\label{sec:exec-summary}

\newpage

\section{Project Description}
\label{sec:project-description}

\subsection{background}
\label{sec:background}
Myopia is one of the most common ocular diseases in human, reaching approximately 25\% of most Western population and more than 80\% in some Asian populations.\cite{rajan98}1 Current research suggests that there is a strong genetic influence upon the development of the condition in addition to extrinsic factors.\cite{zhou99:genes,zhou99:models,schmucker04}2,3,4 As a result of such studies, many ophthalmology researchers have begun to focus on how specific genes affect the development and maintenance of the eye. Mouse has been commonly used as an animal model due to availability, ease of genetic manipulation, and affordability.\cite{schaeffel04}ref 8  However, because precise measurements of the eye dimensions are essential in determining the effect of genes and environmental factors on eye growth and disease, the small size of the mouse eye has been a critical impediment to research in this field.\cite{schaeffel04}ref5 
 
Using the mouse model, researchers need to be able to accurately measure the Anterior-Posterior axis of the small mouse eyeball, whose effective anatomical deformations occur at resolutions of 5 microns.\cite{schaeffel04}ref5  Current techniques for characterizing external eye shape (Table 1 of 4600 final) include manual contact (physical micrometry),\cite{wallman04} ref6 histological sectioning,\cite{schaeffel04}ref5 magnetic resonance imaging (MRI), computed tomography (CT) imaging, positron emission tomography (PET) imaging,\cite{atchison04}ref7 and optical interferometry.\cite{schaeffel04,guggenheim04}ref5,8 Each technique measures the eye but is either too difficult for common use or too inaccurate for precise use. Presently, no method exists for rapidly, easily, and accurately determining the gross exterior shape of the eye at a resolution under 5 μm within a laboratory setting. 

\subsection{Target Device}
\label{sec:target-device}
One current technique for determining eye shape involves manual measurement of the eye dimensions on the micrometer level using a high-precision, high-throughput industrial laser-emitting diode (LED) micrometer. \cite{Nickerson}ref9 (SIA) While other technical methods involve complex optical manipulations, automated measurement and manipulation of an LED micrometer could fill the oft-cited need for a lab bench-sized, automated digitizer which can handle small organic objects such as dissected mouse eyes.\cite{schaeffel04,atchison04,zhou99:genes,zhou99:models} ref8,1,2,3 of project update  The ideal design involves a high-precision articulation frame which locates and manipulates the
measurement plane of an attached micrometer in order to scan over the full geometry of the eye. Collected scanning data is then sent to a computer controller for decoding and used for construction of a digital, three-dimensional (3D) model of the scanned object. Such a device could quickly, precisely, and repeatedly provide the full gross geometric shape of a dissected eye and also provide for software to statistically analyze trends in shape over populations and experimental treatments.

The team has proposed a device which implements complete 3D reconstruction of the eye by scanning over the eye with a laser-emitting diode (LED) micrometer. The device, pictured in FIGUREONE(SIA), consists roughly of three parts including an eye-actuating harness, the micrometer frame, and
the computer interface. The eye harness holds the eye and fixes it rigidly for acceleration without impacting the optical path for the micrometer. It consists of a pair of reverse-action tweezers which are fitted into a cylindrical cuvette filled with a high-viscosity polydimethylsiloxane (PDMS) buffer suspension. When the cuvette is inserted into the frame of the device, it is placed against a corrective lens which corrects for the optical distortion from the cylinder (not shown). The harness then attaches to a motor supplemented by an angular encoder, which can then accelerate the tube while simultaneously keeping high-resolution data on its current theta-position (PICTURES/ADDENDUMS/LABELING OF PARTS?).

The micrometer frame holds the motors and the micrometer heads, which consist of the receiver and transmitter. The heads are attached to a syringe pump frame augmented with another motor/encoder pair to provide linear articulation in the z-direction while maintaining high-resolution position data. Finally, the data from the two encoders and the micrometer is fed into a National Instruments Single-board Reconfigurable Input Output device (sbRIO), which provides real-time feedback and control while collecting data for online reconstruction (PICTURES/ADDENDUMS??). The reconstruction algorithm itself is a modified back-projection algorithm supplemented by cosine interpolation in the z-direction (PICTURES/ADDENDUMS).

\subsection{Target Users and Societal Impact}
\label{sec:users}
The target users of the device will be researchers in ophthalmology, particularly those who conduct studies requiring precise measurements of an eye. An estimated number of 500 researchers working at 70 medical research institutions will benefit from the device\cite{Nickerson}ref9 (SIA), which  will enable researchers to conduct current research easier and faster and introduce new avenues such as scanning techniques through which to study the eye. The improved performance supplied by the device would translate into savings, in terms of both time and money, for both researchers and organizations
that fund them, such as the National Institutes of Health (NIH). This in turn would allow money from the
general public and resources at research institutions to be invested in making discoveries on more fronts,
advancing scientific knowledge at a greater rate. %apparently we want to quantify this last bit how does this make sense?

\section{Engineering Design Specifications}
\label{sec:engin-design-spec}
A complete list of engineering design specifications is shown in Table \ref{tab:eds}.

\subsection{Functional and Customer Requirements}
\label:{sec:func-cust-req}
The customer requirements can be split into three categories: needs, wants, and desires. ``Needs'' are the functional requirements and constitute essential aspects of the device required for proper use, while ``wants'' are non‐essential aspects of the device that would facilitate proper use. ``Desires'' are aspects of the device which, if satisfied, would please the customer beyond proper functioning.  

The most significant feature needed by the device is the ability to take automated measurements of ocular dimensions with a resolution of 0.5 diopters, equivalent to a resolution of 2 $\mu$m for mouse eyes. The critical active sensing components of the device need to have a resolution below 2 $\mu$m in order to ensure precise
measurements at a resolution of 2 $\mu$m. The device also needs to be
able to convert raw measurements into a 3D interactive reconstruction
of the eyeball on a computer interface. As requested by the client,
the device needs to be able to take a complete set of measurements in
under 3 minutes in order to prevent the eye from drying out and
distorting the results. Also, the device needs to be able to fit onto
a typical laboratory bench, into a space of 1m $\times$ 1m $\times$
1m, but does not need to maintain a sterile field.

One ``want'' of the client is that the device has mobility; that is,
it can be lifted and carried from one lab bench to another by an
average healthy researcher from age 20‐50. Therefore, the device
should be lightweight; the weight of the device should not exceed 40
lbs (18 kg).\cite{gross03} To fulfill the mobility want, the size of
the device must also be small. The device must
be able to fit on a typical laboratory bench, in a space of 1m $\times$ 1m $\times$ 1m. The device should also not be cumbersome; the design of the device should be
simple and manageable with few external components.

The customer also wants the device to have a simple user interface with which measurements taken by the device may be transferred onto a connected computer with just 1 click of the mouse. The device should  be reusable for up to 4 years given constant daily use of the device at least 5 times a day, with 1 hour per use.\cite{keyence01} The device should not require any maintenance unless one of the components stops functioning.\cite{keyence01} In addition, the device should cause little to no vibration during operation such that the accuracy of measurements is unaffected. 

Finally, the client desires that the device be aesthetically pleasing, and not make much noise during operation. The noise produced by the device should not exceed the level of 80 dB, the loudness of the average car.\cite{truax09}

\subsection{Engineering Characteristics}
\label{sec:eng-char}
The device needs to be robust enough to withstand the stresses of
everyday lab use. Robustness is defined as the ability to withstand
rough accidental movement such as being dropped from the top of a lab
bench, typically at a height of 3 feet, or shoved sideways on the
bench surface. The device should also be able to hold the weight of
small objects no heavier than 5 pounds, such as lab notebooks, pens,
or a rack of test tubes. The device needs to be able to function in
the wide range of temperatures typically found in the laboratory (22±5
$^\circ$C) and, as mentioned previously, it needs to be able to fit comfortably onto a lab bench withina space of 1m $\times$ 1m $\times$ 1m. The non‐electrical components
of the device should be inert and should not be affected by small
quantities (less than 1 mL) of minimally hazardous solutions such as
PBS,\cite{users_manual} which is typically used to keep the eye
moist. Electrically, the device needs to function with 120V 60Hz AC,
the standard power supply of the United States.

\section{Prototype Development Discussion}
\label{sec:prot-devel-disc}
 
The device design consisted of four main aspects. The frame, eye harness, electrical components, and computer interface were all developed separately. Each aspect of the device underwent several iterations. A prototype development timeline can be seen in TABLETWO-TO-BE-UPLOADED.

\subsection{Introduction}
\label{sec:prior-art}
	The central component of the device is the Keyence LS-7030 LED digital micrometer provided by 
the client. The LED micrometer assumes that light travels between the emitter and receiver in an 
undistorted fashion, and that only the measured object occludes some of the light incident upon the 
receiver, as reviewed in US patent No.6,947,152, High Speed Laser Micrometer.  A prior art and literature search found that among the current techniques, x-ray computed tomography is most relevant to the design solution of the current opportunity because its core measuring component is similar to the LS-7030 micrometer in concept. The principal steps in computed tomography 
include measuring attenuation at different angles, reconstructing a slice via back-projection, and repeat the step for next slice. The design solution was based on this process, which requires angular (theta) and 
translational (z) motion and image reconstruction. 

\subsection{Design Components}
\label{sec:prototype-comp}
	The design solution prototype employs two motors for angular and translational motion. The initial design focused on resolution of the resulting image and intended to move the micrometer both translationally and angularly. However, the team found that this design was hampered by a significant moment of inertia due to the weight of the micrometer. In addition, the client raised concerns about damaging the micrometer, the core piece of the device. 

Given these concerns, the team re-designed the device to rotate the target of interest (the mouse eyeball) and, at the same time, move the micrometer translationally. In accordance with these goals and the requirements of the EDS (TABLE ONE), the design solution consists of the following components: 

\begin{enumerate}
 \item an optical tube with polymer fixate that suspends the mouse eyeball and helps it maintain its overall shape; 
\item a harness that prevents the tube from moving translationally as it is being rotated;
\item motors for angular and translational motion; 
\item physical connector pieces between the rotational motor and the cylindrical tube and between the translational motor and the micrometer;
\item a structural frame work to ensure physical alignment between components; 
\item encoders, in conjunction with a National Instruments sbRIO board, which manipulate the motors and transfer data in between the motors to the computer;  
\item an image reconstruction algorithm created in the LabVIEW Development System;
\item and a user-centered computer interface. 
\end{enumerate}


[frame]
	The device is set up so that the axis of rotation of the theta motor, the eyeball, and the structures 
that support the eyeball is along the same axis. The micrometer heads are attached to their base, 
connecting to a block whose position is controled by the z-motor through a threaded shaft, in such a way 
that the plane of measurement is perpendicular to the axis of rotation. To achieve good alignment, the 
solution uses specially designed pieces to hold components in place. The solution has a base where the 
mechanical components sit on. A frame work, consisting of a top, a bottom, and two legs that connect and 
support the top piece to the bottom piece, attaches to the base with bolts and nuts. The theta-motor sits 
press-fitted in a designed cavity and defines the axis of rotation. The top piece has a hole that is concentric 
to the axis of rotation and serves to maintain alignment of a set of core pieces that holds the eyeball. 

[harness and fixate]
	The core pieces consist of a clear tubing, top and bottom caps, holder on the top cap that holds 
the eyeball in the tubing, and a connector between bottom cap and theta-motor. All of the core pieces are 
cylindrical and have their centers on the axis of rotation. The bottom cap is detachable from the connector 
to theta-motor so the tubing can leave the device to be loaded with new eyeball. The tubing will be loaded 
with eyeball and tissue fixate that keeps the shape of the eyeball. Out side of the core, there will be optical 
components that corrects the path of light that's distorted by the tube and fixate solution.

[micrometers]
	On the opposite side of the two support legs, the guiding and driving shafts and z-motor for 
translational motion sits in simple aluminum housing fixed to the base. A connector block mounted on the 
shafts connects to the base of the micrometer so that the z-motor can drive the micrometer to different z-
position. A computer connects to the motors to control them and reads data from the micrometer for image 
reconstruction. 
	
[optics somewhere...]
[image reconstruction algorithm]

[prototype construction]
	Construction of the prototype employs Zprinter® 310 (inkjet) Printer with special starch powder 
and the Stratasys(Registered) Dimension 1200 FDM (Fused deposition modeling) Printer with ABS 
(acrylonitrile butadiene styrene) material to print the structural framework for review and testing.

[Iteration]
	The first draft of the prototype structural framework came from the Zprinter and showed that the 
structures are sturdy enough to not vibrate much. However, since he Zprinter does not have good enough 
resolution for tight connection and precise position, the FDM Printer is chosen to print the structures to be 
used in final prototype. The second draft used the same blueprint but was printed with the FDM printer. The 
second draft is lighter and has better precision and fitting between parts. It is also strong and does not 
vibrate much.  Third draft is being printed now........
	The first z-motor was was 0.6A mounted on a modified microscope stage. It did not have enough 
power to drive the micrometer and was switched out with a 2A stepper motor on syringe pump.


\section{Performance Testing and Results}
\label{sec:perf-test-results}

Two kinds of performance testing is performed: Functionality Tests and Unit Tests. Functionality Tests 
verifies whether the whole device functions as intended. Basically, the device will measure a known 
dimension test object. The collected data will be compared to a gold standard digital mesh of the test 
object. The deviation quantifies the errors produced by the whole device. Success of this test implies that 
the device is fully functional to the needs of the final consumer, though it alone is not sufficient for testing 
purposes. Without the supporting evidence of the unit tests, it is difficult to understand either why this test 
fails if it does or if the device will perform consistently with repetition.

NO TESTING DONE ON  THIS YET???

If functionality test fail, it is evident that any combination of the parts of the device maybe the source of 
failure. Unit tests dissect the problem by functional subunit. There are several subunits that need testing: 
Eye harness, Frame, and Controler. 

Eye harness testing involves optical correction testing

NO TESTING DONE ON  THIS YET???

Frame testing includes structural stability testing and actuator function testing.

NO TESTING DONE ON  THIS YET???

Controler testing consists of I/O functionality testing and Data reconstruction testing.

NO TESTING DONE ON  THIS YET???


we've done nothing according to our protocol???



\section{Final Design Solution}
\label{sec:final-design-solut}

\section{Discussion}
\label{sec:discussion}

\section{Conclusion and Recommendations}
\label{sec:concl-recomm}

\newpage
\addcontentsline{toc}{section}{References}
\bibliographystyle{unsrt}
\bibliography{../tex/bibl}

\end{document}
%%% Local Variables: 
%%% mode: latex
%%% TeX-master: t
%%% End: 
