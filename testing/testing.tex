%%% Local Variables: 
%%% mode: latex
%%% TeX-master: t
%%% End: 
\documentclass{article}
\usepackage{../tex/mysty}
\begin{document}

\maketitlepage{Testing and Verification Protocol}{Joseph Abrahamson}

\setcounter{tocdepth}{2}
\tableofcontents
\newpage

\section*{Executive summary}
\label{sec:exec-summary}


\newpage

\section{Testing methods}
\label{sec:testing-protocols}

A prototype device is considered successful if it fufills the needs
outlined in the Engineering Design Specifications. For our device,
this can be simplified to four testing concerns which combine
effectively to ensure a successful device. These concerns will be
referred to throughout this document.

\paragraph{Fixation} The eye to be measured must fit into its harness
and thus be appropriately fixed. It should act as a rigid body under
the forces (and pertubations from) constant velocity during
measurement. It should remain fixed in location relative to its
harness and not slip or wobble.

\paragraph{Articulation and positioning} The harness must be
articulated accurately and precisely by the frame
motors. Additionally, the position of the harness must be accurately
measured by the motors' encoders in order to perform the
reconstruction.

\paragraph{Recording} The micrometer must be able to read the eye
without hinderance or distortion from the harness.

\paragraph{Noise minimization} The noise induced from eye, harness,
and frame vibrations should not total to overwhelm the desired
accuracy bounds. Additionally, the reconstruction algorithm should not
transform small signal noise to large disortions at any point.

A device implementation which can achieve these four ends will succeed
in meeting the design specifications and thus be a strong choice for a
final prototype.

\subsection{Performing the tests}
\label{sec:hierarchical-testing}

Since all device implementations will contain a number of subsystems,
it is necessary to test each one in isolation as a set of unit
tests. The necessary assumptions for a subsystem to act as a unit is
that each component is fault-free and no components interact to cause
a failure. For our device, this demands that each motor and frame
component is individually tested and then in combination. This forces
a necessary hierarchy and ordering on the tests: systems are only
tested once all of their dependent systems have individually passed.

The test protocols in the rest of this document are presented with
respect to this dependency ordering.

\subsection{Data acquisition and storage}
\label{sec:data-acqu-stor}

The data obtained during these tests is never sensitive but may very
easily be extremely high volume. This implies that storage need not be
secure, but some preproccessing will be necessary. Each test sequence
will be labeled by date, prototype ID and necessary preprocessing and
sufficient summary statistics will be presented in each section as
necessary as the \emph{stored test data}.

\subsection{Statistics}
\label{sec:statistics}

The tests considered follow two general patterns: continuous error
propagation and binary (pass/fail) decisions. Each test will be
labeled by the form of statistic needed to evaluate test certainty.

\subsubsection{Continuous}
\label{sec:continuous}

These tests result in an uncertain measurement of the amount of
physical noise induced by a given subsystem. Since physical noise is
linear and has infinite support, we represent noise as the variance in
a Gaussian term. Since overall noise must be less than \SI{2}{\micro
  m} in order to ensure sufficient accuracy, these tests only fail if
we are less than 95\% certain that the noise induced by this level is
sub-\SI{2}{\micro m}.

The geometry of the device is used to combine errors in larger
systems. For instance, the noise at a certain subsystem could be the
weighted sum of the noises generated at its component sublayers. In
general, the expectation of total error is monotonically increasing in
more complex subsystems and the uncertainty is additive therefore
exceeding \SI{2}{\micro m} at any point constitutes a failure and
justifies discontinuing tests.

\subsubsection{Binary}
\label{sec:binary}

These tests can result as either a certain or probabilistic pass or
failure. For instance, the type of support polymer used in the eye
harness could either obstruct the LED measurement beam (fail) or
permit it (pass). Since this is purely a function of very consistent
physical processes --- the absorbance spectrum of the polymer --- a
single test is sufficient for prototyping purposes. Other binary tests
such as the presence slippage between the harness and the articulating
motors may be better modeled via a full failure analysis.

Failure analysis is interested in the average time between failures or, in the event of catastrophic failure, the average lifetime. These are roughly equivalent provided the chance of failure ($p$) is low and can both be represented as a discrete count $m = 1/p$. In general, to be 95\% certain that the average failure time is less than some $m$ you must

\subsection{Presentation}
\label{sec:presentation}

Due to the dependent nature of the full testing protocol, the
progress of a given prototype through the full verification protocal
can be easily visualized. Additionally, since both expected value and
uncertainty about the induced noise are maintained and combined, at
any given point the distribution of the noise can be graphically
presented to indicate chance of a sufficiently accurate measurement.

An example summary table is presented in Figure
\ref{fig:result-summary}.

\section{Testing Protocols}
\label{sec:protocols}

%
% See notes.org
%

\section{Addendums}
\label{sec:addendums}

\newpage
\addcontentsline{toc}{section}{References}
\bibliographystyle{unsrt}
\bibliography{../tex/bibl}

\end{document}
