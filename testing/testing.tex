%%% Local Variables: 
%%% mode: latex
%%% TeX-master: t
%%% End: 
\documentclass{article}
\usepackage{../tex/mysty}
\begin{document}

\maketitlepage{Testing and Verification Protocol}{Joseph Abrahamson}

\setcounter{tocdepth}{2}
\tableofcontents
\newpage

\listoftables
\listoffigures
\newpage


\section*{Executive Summary}
\label{sec:exec-summary}


\newpage

\section{Testing Methods}
\label{sec:testing-protocols}

\subsection{Data Acquisition and Storage}
\label{sec:data-acqu-stor}

\subsection{Statistics}
\label{sec:statistics}

\subsection{Presentation}
\label{sec:presentation}

\section{Testing Protocols}
\label{sec:protocols}

Ginger: physical structure of the harness (it must support the eye, keep it safe, rotate cleanly, fit into frame, etc etc)

Name of test: Tweezer support and eyeballs
Importance of test (vital for operation ~ interesting to know): Tests whether the optic nerve is held tightly enough by the tweezers to prevent unnecessary 
eyeball movement/rotation 
Purpose and background: The reverse action tweezers have been modified to fit into the physical top cap of the piece. This test tests whether the cap has been made
properly, that is, whether the tweezers are still functional within the cap (do they hold the eye tightly enough - but not too tight that the optic nerve breaks when
motor is spun)
Functionality this ties in to (maybe even tie it to an EDS if that's pertinent): don't think this fits with any EDS. Functionality: effect of cap on tweezers...does the
cap actually work???
How to test (measurement, outline of a protocol)
1. spin the motor - incrementally increase speed
2. scan eyeball at same time
3. note any distorted images or extreme movement of eyeball (like is the eye hitting the side of the tube)
The set of test metrics (possibly just one, I suppose): speed at which tweezers drop the eyeball
Possible test results (what does a successful test look like, what does a failed test look like, miserably failed test?): miserable failure - eyeball breaks off from optic nerve; 
eyeball gets dropped
(with this one, quantify: exactly how bad might we expect a failure to be? how separable are the test metrics in a good and a failed device?)
ummm...it qualifies as failure if the eyeball breaks off from nerve when the motor is spinning at a speed range that we expect to use for the actual device. If it breaks off at 
an extreme speed not to be used to operate the device, doesn't count as failure. 
Exactly how bad do we expect a failure to be...I'm not sure...

Any other notes that would help me write the details: ? 


Name of test: Bottom cap/motor
Importance: Does the bottom cap connect and fit into the motor and not fall off the motor when the motor spins really fast - so basically is the bottom cap held tightly enough
by the motor
Purpose/background: This is a physical check to make sure that the motor and the cap are connecting properly and that the cap is rotating at same speed as motor
Functionality: same as purpose????
Protocol: 
1. spin the motor - incrementally increase speed
2. find speed at which bottom cap disconnects from motor
Test Metrics: the range of #2 above
Possible test results: successful test - stays on, failed test - begins to come loose, miserably failed - flies across the room and kills somebody or shatters into the wall epic fail
an epic failure would be if the cap disconnects at a speed which we expect to use in the actual device (a low speed)
Notes: sorry this test is made of fail 



Name of test: Tweezers/top cap
Importance: Do the tweezers fit
Purpose/background: There are two components. Is the hole big enough on the top cap or is the hole designed properly to support tweezers
Functionality: don't think this fits into EDS. 
Protocol: 
1. stick the tweezers in the hole
2. do they fall through? 
3. can you open and close the tweezers by a large enough distance that they can still grasp the eyeball?
Test Metrics: how much can you open and close the tweezers? see #3 on protocol. Is there enough leeway (not sure how much numerically) to grasp eyeball and support eyeball when
tweezer is inserted into the top cap?
Possible test results: 
FAILURE: they don't fit - they fall through the hole or cannot be opened
SUCCESS: the tweezers are still functional inside the hole
MODERATE: tweezers can open and close but not enough to grasp eyeball
Notes:


Name of test: Tube/caps/silicon
Importance: do the caps and silicon seal the tube
Purpose/background: are the caps going to fall off with high velocity motor spin?
Functionality: same as purpose?
Protocol: 
1. spin motor and incrementally increase speed
2. do caps fall off? (does Si seal the caps to the tube as it should?)
Test Metrics: not sure of any numbers  - range of distance that caps detach from the tube
Possible test results: does cap completely detach from tube or just a little bit at high-velocity motor speed (or lower motor speeds that we use to perate the device)
Notes:


Name of test: fluid inside tube 
Importance: Is fluid transparent enough for scanning to be successful
Purpose/background: eye will be fixed in polymer solution, so how much does sol'n affect the scan? can eyeball be detected effectively or does solution completely block transmittance of light?
Functionality: 
Protocol: 
1. fill fluid in tube
2. scan eyeball
Test Metrics: ???
Possible test results: eyeball detected but distorted by the solution; eyeball detected and not distorted at all by solution; eyeball cannot be detected at all because solution is too opaque (also the tube we use may have an effect)
Notes: 

I'll do the final copy-editing as usual -Ginger 


Shuyen's part--------------------v----------------------------
Name: harness frame strength test
Importance: good to know that it’s strong, meeting robustness EDS
Purpose: making sure the frame is strong enough to hold things in place
Functionality: Probably related to robustness – things must survive some amount of force
How to test: subject individual pieces to different loading: tensile and bending
          Take one piece; hold it in place with clamp(s)
          Load weight on it: 30g, 50g, 100g, 300g, (1kg?) at different position
          Change the orientation of the piece and load again
          Repeat for different pieces
Test metrics: Orientation of piece
Amount of weight loaded, 
           Position of loading
           Breaking (y/n)
Possible results: Good: no breaking
              Bad: anything breaks

Name: harness frame assembly failure test 
Importance: meeting EDS
Purpose: To verify the specification of ability to sustain 3-ft drop impact
Functionality: see above
How to test: Assemble frame structure onto base
          Drop the assembly from 3-ft height to concrete floor
          Examine frame for any material failure (i.e. fracture, breaking) or detachment from base or from each other
           Repeat several (5-10?) times and record the number of repeat and the result of each repeat
Test metrics: number of repeat (values are from 1-10)
                   Fracture (y/n)
                   Breaking (y/n)
                   Detachment from base (y/n)
                   Loose connection between frame pieces (y/n) 
Possible test results: good result: all the y/n question receives an “n” (no fracture, breaking, detachment etc)
                 Bad result: any fracture, breaking, detachment 

Name: harness frame stiffness test
Importance: meeting stability EDS
Purpose: make sure the frame can hold things in their relative position without much deviation
Functionality: probably related to stability and accurate measurement – things need to be in the right places and not move too much
How to test: see harness frame strength test
           Note significant bending (visible to the eye, y/n)
           Measure deflection distance, also record bending direction
Test metrics: same as frame strength test
Significant bending (y/n)
           Deflection (length unit)
Possible results: Good: no significant bending at all
              Bad: Significant bending, especially at low weight (low force)

Name: Thermal expansion test
Importance: good to know? May need to meet EDS 
Purpose: to verify that the device will work in the temperature range of 22±5 °C
Functionality: structural integrity at room temperature
How to test: Assemble the device at 17 °C
          Heat the device up to 27 °C (use a convection incubator or something)
          Examine device for fracture  
          Assemble the device at 27°C
          Cool it down to 17°C (refrigerator?)
          Examine device for loose connection or detachment
Test Metrics: fracture with increased T (y/n)
           Loose connection with decreased T (y/n)
Possible results: good: no fracture or loose parts
              Bad: fracture/loose parts, or not able to assemble when temperature is higher

\section{Addendums}
\label{sec:addendums}

\newpage
\addcontentsline{toc}{section}{References}
\bibliographystyle{unsrt}
\bibliography{../tex/bibl}

\end{document}
